\newcommand{\curs}{\mathcal}                                                % raccourci perso pour lettres en écriture anglaise
\newcommand{\dessin}{\includegraphics}                                      % raccourci perso pour insérer les graphiques

\newcommand{\be}{\begin{enumerate}}
\newcommand{\ee}{\end{enumerate}}

%-------------Lignes de points--------------- \dline et answerline
\newcommand\dline[1][6cm]{\makebox[#1]{\dotfill}}
\makeatletter
\newcommand\answerline{\@ifnextchar[%]
  \answerlinetowidth\answerlinetoeol}
\newcommand\answerlinetowidth[1][0pt]{\hbox to #1{\leaders\hbox to \answerdotsep{\hss.\hss}\hfill}}
\newcommand\answerlinetoeol{\leaders\hbox to \answerdotsep{\hss.\hss}\hfill\strut}
\newcommand\answerdotsep{6pt}
\makeatother
%--------------------------------------------------

\newcommand{\g}[1]{\og #1 \fg}

\setlength{\columnseprule}{0.5pt}
% \usepackage{hyperref}
% \hypersetup{
%     colorlinks=true,
%     linkcolor=teal,
%     urlcolor=blue,
%     pdftitle={IE1 Droites},
%     pdfpagemode=FullScreen,
%     pdfauthor={Réginald Bayle},
% }

%Ajout :
\def\hexagonsize{0.1cm}
\pgfdeclarepatternformonly
  {hexagons}% name
  {\pgfpointorigin}% lower left
  {\pgfpoint{3*\hexagonsize}{0.866025*2*\hexagonsize}}%  upper right
  {\pgfpoint{3*\hexagonsize}{0.866025*2*\hexagonsize}}%  tile size
  {% shape description
   \pgfsetlinewidth{0.4pt}
   \pgftransformshift{\pgfpoint{0mm}{0.866025*\hexagonsize}}
   \pgfpathmoveto{\pgfpoint{0mm}{0mm}}
   \pgfpathlineto{\pgfpoint{0.5*\hexagonsize}{0mm}}
   \pgfpathlineto{\pgfpoint{\hexagonsize}{-0.866025*\hexagonsize}}
   \pgfpathlineto{\pgfpoint{2*\hexagonsize}{-0.866025*\hexagonsize}}
   \pgfpathlineto{\pgfpoint{2.5*\hexagonsize}{0mm}}
   \pgfpathlineto{\pgfpoint{3*\hexagonsize+0.2mm}{0mm}}
   \pgfpathmoveto{\pgfpoint{0.5*\hexagonsize}{0mm}}
   \pgfpathlineto{\pgfpoint{\hexagonsize}{0.866025*\hexagonsize}}
   \pgfpathlineto{\pgfpoint{2*\hexagonsize}{0.866025*\hexagonsize}}
   \pgfpathlineto{\pgfpoint{2.5*\hexagonsize}{0mm}}
   \pgfusepath{stroke}
  }
%Exemple : \fill[pattern=hexagons] (0,0) circle (3cm);

\usetikzlibrary{patterns.meta}%Plus d'options de patterns

%%% Ajout de mylines :
\tikzdeclarepattern{
  name=mylines,
  parameters={
      \pgfkeysvalueof{/pgf/pattern keys/size},
      \pgfkeysvalueof{/pgf/pattern keys/angle},
      \pgfkeysvalueof{/pgf/pattern keys/line width},
  },
  bounding box={
    (0,-0.5*\pgfkeysvalueof{/pgf/pattern keys/line width}) and
    (\pgfkeysvalueof{/pgf/pattern keys/size},
     0.5*\pgfkeysvalueof{/pgf/pattern keys/line width})},
  tile size={(\pgfkeysvalueof{/pgf/pattern keys/size},
              \pgfkeysvalueof{/pgf/pattern keys/size})},
  tile transformation={rotate=\pgfkeysvalueof{/pgf/pattern keys/angle}},
  defaults={
    size/.initial=5pt,
    angle/.initial=45,
    line width/.initial=.4pt,
}, code={
\draw [line width=\pgfkeysvalueof{/pgf/pattern keys/line width}] (0,0) -- (\pgfkeysvalueof{/pgf/pattern keys/size},0);
}, }
%Exemple : \draw[pattern={mylines[size=10pt,line width=.8pt,angle=10]},pattern color=red] (0,0) rectangle ++(2,2);

%%% Ajout de hatch:
\pgfdeclarepattern{
name=hatch,
parameters={\hatchsize,\hatchangle,\hatchlinewidth},
bottom left={\pgfpoint{-.1pt}{-.1pt}},
top right={\pgfpoint{\hatchsize+.1pt}{\hatchsize+.1pt}},
tile size={\pgfpoint{\hatchsize}{\hatchsize}},
tile transformation={\pgftransformrotate{\hatchangle}},
code={
\pgfsetlinewidth{\hatchlinewidth}
\pgfpathmoveto{\pgfpoint{-.1pt}{-.1pt}}
\pgfpathlineto{\pgfpoint{\hatchsize+.1pt}{\hatchsize+.1pt}}
\pgfpathmoveto{\pgfpoint{-.1pt}{\hatchsize+.1pt}}
\pgfpathlineto{\pgfpoint{\hatchsize+.1pt}{-.1pt}}
\pgfusepath{stroke}
}
}
\tikzset{
hatch size/.store in=\hatchsize,
hatch angle/.store in=\hatchangle,
hatch line width/.store in=\hatchlinewidth,
hatch size=5pt,
hatch angle=0pt,
hatch line width=.5pt,
}
%Exemple : \filldraw[pattern color=gray!70,pattern=hatch,hatch size=10pt](3.2,0) rectangle (3.8,12)

\newtcbtheorem{theo}%
  {Théorème}{colback=red!5!white,colframe=red!50!black,title=My nice heading}{theorem}
%\begin{theo}{Like and Subscribe}{likesub}
%For every person $x$, $x$ should like and subscribe.
%\end{theo}
%\newcommand{\degree}{\ensuremath{^\circ}}
%\DeclareMathAccent{\wideparen}{\mathord}{largesymbols}{"F3}
%\def\wideparen#1{\overset{\;\rotatebox{90}{)}}{#1}}

%----------------------------------------

% \newcommand{\rul}[1]{\textcolor{red}{\underbar{\textcolor{black}{#1}}}}
% \newcommand{\bul}[1]{\textcolor{blue}{\underbar{\textcolor{black}{#1}}}}
% \newcommand{\vul}[1]{\textcolor{violet}{\underbar{\textcolor{black}{#1}}}}
% \newcommand{\gul}[1]{\textcolor{green!60!black}{\underbar{\textcolor{black}{#1}}}}
% \newcommand{\nul}[1]{\underbar{#1}}
% \newcommand{\attention}{\includegraphics[scale=0.2]{images/attention.png}\textcolor{red}{Attention} : }

% \newcommand{\rul}[1]{\textbf{\textsc{#1}}}
% \newcommand{\bul}[1]{\textsc{#1}}
% \newcommand{\gul}[1]{\textbf{#1}}
% \newcommand{\vul}[1]{\textsc{\textit{#1}}}

%%%%%%%%%%%% Pour avoir l'arc pour les plus que demi-cercle
\makeatletter
\def\widebreve{\mathpalette\wide@breve}
\def\wide@breve#1#2{\sbox\z@{$#1#2$}%
     \mathop{\vbox{\m@th\ialign{##\crcr
\kern0.08em\brevefill#1{0.8\wd\z@}\crcr\noalign{\nointerlineskip}%
                    $\hss#1#2\hss$\crcr}}}\limits}
\def\brevefill#1#2{$\m@th\sbox\tw@{$#1($}%
  \hss\resizebox{#2}{\wd\tw@}{\rotatebox[origin=c]{90}{\upshape(}}\hss$}
\makeatletter

%%%%%%%%%%% Pour avoir l'arc pour les moints que demi-cercle
\makeatletter
\def\wideparen{\mathpalette\wide@paren}
\def\wide@paren#1#2{\sbox\z@{$#1#2$}%
     \mathop{\vbox{\m@th\ialign{##\crcr
\kern0.08em\parenfill#1{0.8\wd\z@}\crcr\noalign{\nointerlineskip}%
                    $\hss#1#2\hss$\crcr}}}\limits}
\def\parenfill#1#2{$\m@th\sbox\tw@{$#1($}%
  \hss\resizebox{#2}{\wd\tw@}{\rotatebox[origin=c]{90}{\upshape)}}\hss$}
\makeatletter
%%%%%%%%%%% Pour avoir le chapeaux de angles rentrants
\makeatletter
\DeclareRobustCommand\widecheck[1]{{\mathpalette\@widecheck{#1}}}
\def\@widecheck#1#2{%
    \setbox\z@\hbox{\m@th$#1#2$}%
    \setbox\tw@\hbox{\m@th$#1%
       \widehat{%
          \vrule\@width\z@\@height\ht\z@
          \vrule\@height\z@\@width\wd\z@}$}%
    \dp\tw@-\ht\z@
    \@tempdima\ht\z@ \advance\@tempdima2\ht\tw@ \divide\@tempdima\thr@@
    \setbox\tw@\hbox{%
       \raise\@tempdima\hbox{\scalebox{1}[-1]{\lower\@tempdima\box
\tw@}}}%
    {\ooalign{\box\tw@ \cr \box\z@}}}
\makeatother
%%%%%%%%%%%%%%%%%%% Arc de cercle centré en 2 %%%%% Fonctionnent pas et je ne sais pas pourquoi
%\def\angle[#1](#2)(#3:#4:#5){\filldraw [#1,domain=#3:#4, fill opacity=0.2] (#2)--plot ((#2)+({#5*cos(\x)},{#5*sin(\x)})--(#2);}
%\filldraw [green!70!black,domain=105:435, fill opacity=0.2] (0,0) -- plot ({0.8*cos(\x)}, {0.8*sin(\x)})--(0,0);
%\angle[green!70!black](0,0)(105:435:0.8);
%\def\centerarc[#1](#2)(#3:#4:#5){\draw[#1]([shift=(#3:#5)]#2) arc (#3:#4:#5);}%
%Syntax: [draw options] (center) (initial angle:final angle:radius)
    
%\def\centerarc[#1](#2)(#3:#4:#5){ \draw[#1] ($(#2)+({#5*cos(#3)},{#5*sin(#3)})$) arc (#3:#4:#5) -- (#2); }
\def\angle[#1](#2)(#3,#4,#5){\filldraw [domain=#3:#4, fill opacity=0.2,#1] (#2)--($(#2)+({#5*cos(#3)},{#5*sin(#3)})$) arc (#3:#4:#5)--(#2);}
\def\centerarc[#1](#2)(#3,#4,#5){\draw[#1] ($(#2)+({#5*cos(#3)},{#5*sin(#3)})$) arc (#3:#4:#5); }
%{\draw [#1,domain=#3:#4] plot ((#2)+({#5*cos(\x)}, {#5*sin(\x)}));}
    % Syntax: [draw options] (center) (initial angle:final angle:radius)
%Exemple : \centerarc[red,thick](0,0)(5:85:1)
%---------------------- Pour les flèches pour les tableaux de proportionnalité
\newcommand{\marqueur}[1]{\tikz[remember picture, overlay]\node[xshift=-0.5ex,yshift=0.4ex] (#1){} ;}
\newcommand{\markh}[1]{\tikz[remember picture, overlay]\node[xshift=-0.5ex, yshift=0.5em] (#1){} ;}
\newcommand{\markb}[1]{\tikz[remember picture, overlay]\node[xshift=-0.5ex] (#1){} ;}

\newcommand{\fleche}[5]{\tikz [remember picture, overlay]\draw (#1) edge[min distance=1.5em,bend right=#3,-stealth] node[midway, #4]{#5} (#2);}
%----------------------------------------------------------
%---Convention française de 3D :
% \tikzset{math3d/.style=
% {x= {(-0.353cm,-0.353cm)}, z={(0cm,1cm)},y={(1cm,0cm)}}}
%----
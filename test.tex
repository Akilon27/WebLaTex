\documentclass[a4paper,12pt]{article}

\usepackage[utf8]{inputenc}
\usepackage[T1]{fontenc}
\usepackage[french]{babel}
\frenchbsetup{StandardItemLabels} % pour conserver les puces lors des énumérations
%\usepackage[babel=true,kerning=true]{microtype}
\usepackage[margin=1cm]{geometry}  %permet le réglage des marges
%\geometry{hmargin=1cm,vmargin=1cm} %réglage des marges

%%%%%%%%%%%%%%%%%%%%%%%%%%%%%%%%
% \usepackage{sansmath}
% \sansmath

%%%%%%%%%%%%%%%%%%%%%%%%
%---------Packages MathPix


%\usepackage[export]{adjustbox} %This package allows to adjust general (LA)TEX material in several ways using a key=value interface. It got inspired by the interface of \includegraphics from the graphicx package. This package also loads the trimclip package which code was once included in this package.
\usepackage{bm}%Pour avoir des formules mathématiques en gras
%\usepackage[version=4]{mhchem}
%\usepackage{stmaryrd}
%--------------------------

\usepackage{parskip} %évite que chaque nouvelle ligne soit indentée
\usepackage{setspace}
%\usepackage{xcolor}%[usenames, dvipsnames]
\usepackage{color,colortbl}
\usepackage[table]{xcolor}
%\usepackage{xspace}%va détecter lors de la compilation si une espace doit être ajoutée ou pas à la fin de la macro. On saisira donc : \newcommand\IP{imaginaires purs\xspace}
%\usepackage{picins}%Pour mettre facilement des images à droite
\usepackage{wrapfig}%Un peu la même chose
\setlength{\columnsep}{1cm}
\usepackage[overload]{textcase} 
\usepackage{dashundergaps} %Pour faire des cours à trous
\dashundergapssetup{dot,teacher-gap-format=dot,gap-format-adjust=false,
gap-numbers=false,gap-widen,gap-extend-minimum=30pt,%ajoute x pt au minimum
gap-extend-percent=40,%ajoute x % The result is compared to gap-extend-minimum and the larger of the two is used
%gap-font=\Large,%\bfseries \itshape 
%gap-number-format=\textnormal{ ^{(\thegapnumber)}},
%gap-number-format = \,\textsuperscript{\normalfont(\thegapnumber)},
%display-total-gaps,
}

\usepackage{siunitx}
\sisetup{locale = FR}
%%% Donne accès à la commande \num
%Exemples : \num{.34}   \num{3e8}

%\usepackage[np]{numprint}  %Autre paquet qui gère les espaces dans l'écriture des nombres

%\abovedisplayskip=1ex    Pour changer l'espacement au-dessus des formules mathématiques \[ \]... Il y a aussi \belowdisplayskip ainsi que \abovedisplayshortskip et  \belowdisplayshortskip si la ligne qui PRECEDE est trop courte !

%Assez des salamis, je passe au jambon - Je fais un carnage si ce car nage car je nage, moi, Karnaj !
%Le comble pour un professeur de mathématique ? Mourir dans l’exercice de ses fonctions.

%$\circlearrowleft$ $\circlearrowright$
%$\curvearrowleft$ $\curvearrowright$

\usepackage{cancel}
%$x^2-\cancel{x}+\cancel{x}-1=x^2-1$        $\cancel{A}+\bcancel{B}+\xcancel{C}
%---------------------------
%%%%%%%%%%%%%%%%%%%%%%%%%%%%%%%%%%%%%%%%%%%%%%
\usepackage{amsmath, amsthm, amssymb} % nécessaire pour taper des maths

%%%%%%%%%%%%%%%%%%%%%%%%%%%%%%%%%%%%%%%%%%%%%%%%%%%%%%%
%\usepackage{lscape}  % pour le format paysage
\usepackage{pifont}  % pour les symboles ding
\usepackage{enumerate}  %Permet de renuméroter différement les questions
%\usepackage{esvect}  % permet de faire des beaux vecteurs
\usepackage{multicol}  % permet de fusionner des colonnes
\usepackage{multirow}   % permet de fusionner des lignes
%\usepackage{colortbl}  %couleur dans les tableaux
\usepackage{tabularx}  % permet des tableaux plus compliqués mais pas simple à utiliser.
\usepackage{diagbox}%---Des cellules en diagonales pour tableaux à doubles entrée
\usepackage{array} %,makecell
% \newcolumntype{R}[1]{>{\raggedleft\arraybackslash }b{#1}}  % permet des cellules alignées à droite de taille fixée : écrire L{3cm} par exemple. Ne convient pas à une fusion
% \newcolumntype{L}[1]{>{\raggedright\arraybackslash }b{#1}} % même chose à gauche
\newcolumntype{C}[1]{ >{\centering\arraybackslash} m{#1}}   % même chose au centre
\newcolumntype{T}[1]{ >{\centering\arraybackslash\columncolor{teal!20}} m{#1}} %Avec couleur teal pour les titres de lignes

%\setcellgapes{1pt}     %ajouter  \setcellgapes{4pt} avant un tableau avec des formules mathématiques hautes pour que cela ne touche pas le bord de la cellule
%\makegapedcells
\usepackage{graphicx} % pour l'insersion d'images, incompatible avec xcolor
%\usepackage{caption} % pour des titres dans les tableaux flottants, pas forcément utile
%\usepackage{variations} % permet des supers tableaux de variations
\usepackage{frcursive}  % permet d'écrire comme un écolier, c'est très joli
\everymath{\displaystyle} %pour que les maths soient à la bonne taille (limite, intégrale)
%%%%%%%%%%%%%%%%%%%GrindEQ
\usepackage{txfonts}
\usepackage{mathdots}
\usepackage[classicReIm]{kpfonts}
%%%%%%%%%%%%%%%%%%%%%%%%%%%%%%%

\usepackage{marvosym}   %c'est pour le symbol euro : code \EUR{}
%\usepackage{pst-3dplot}  % cela fait des dessins en 2D mais je ne maitrise pas bien

\usepackage{mathrsfs}
%\usepackage{yhmath}%Pour les arcs de cercles mais des conflits : mieux ci-dessous, redéfini
%\usepackage{tikz}
\usepackage{tkz-euclide}%,pgflibraryshapes
% \usepackage{pgfplots}
\usetikzlibrary{calc,through}
\usetikzlibrary{arrows}
\usetikzlibrary[patterns]
%Pattern names : horizontal lines, vertical lines, north east lines, north west lines, grid, crosshatch	dots, fivepointed stars, sixpointed stars, bricks, checkerboard

\usetikzlibrary{positioning}
%\usepackage{tkz-tab}
%\usepackage{forest}%Pour les arbres de probabilités
% \usetikzlibrary{shadows}%Pour les ombres Tikz
%\usepackage{tikz-qtree}% Autre paquet pour les arbres
%\usepackage{fancyhdr,txfonts,pxfonts}
\usepackage[most]{tcolorbox}
%----------------Pour les paquets
\newcommand{\curs}{\mathcal}                                                % raccourci perso pour lettres en écriture anglaise
\newcommand{\dessin}{\includegraphics}                                      % raccourci perso pour insérer les graphiques

\newcommand{\be}{\begin{enumerate}}
\newcommand{\ee}{\end{enumerate}}

%-------------Lignes de points--------------- \dline et answerline
\newcommand\dline[1][6cm]{\makebox[#1]{\dotfill}}
\makeatletter
\newcommand\answerline{\@ifnextchar[%]
  \answerlinetowidth\answerlinetoeol}
\newcommand\answerlinetowidth[1][0pt]{\hbox to #1{\leaders\hbox to \answerdotsep{\hss.\hss}\hfill}}
\newcommand\answerlinetoeol{\leaders\hbox to \answerdotsep{\hss.\hss}\hfill\strut}
\newcommand\answerdotsep{6pt}
\makeatother
%--------------------------------------------------

\newcommand{\g}[1]{\og #1 \fg}

\setlength{\columnseprule}{0.5pt}
% \usepackage{hyperref}
% \hypersetup{
%     colorlinks=true,
%     linkcolor=teal,
%     urlcolor=blue,
%     pdftitle={IE1 Droites},
%     pdfpagemode=FullScreen,
%     pdfauthor={Réginald Bayle},
% }

%Ajout :
\def\hexagonsize{0.1cm}
\pgfdeclarepatternformonly
  {hexagons}% name
  {\pgfpointorigin}% lower left
  {\pgfpoint{3*\hexagonsize}{0.866025*2*\hexagonsize}}%  upper right
  {\pgfpoint{3*\hexagonsize}{0.866025*2*\hexagonsize}}%  tile size
  {% shape description
   \pgfsetlinewidth{0.4pt}
   \pgftransformshift{\pgfpoint{0mm}{0.866025*\hexagonsize}}
   \pgfpathmoveto{\pgfpoint{0mm}{0mm}}
   \pgfpathlineto{\pgfpoint{0.5*\hexagonsize}{0mm}}
   \pgfpathlineto{\pgfpoint{\hexagonsize}{-0.866025*\hexagonsize}}
   \pgfpathlineto{\pgfpoint{2*\hexagonsize}{-0.866025*\hexagonsize}}
   \pgfpathlineto{\pgfpoint{2.5*\hexagonsize}{0mm}}
   \pgfpathlineto{\pgfpoint{3*\hexagonsize+0.2mm}{0mm}}
   \pgfpathmoveto{\pgfpoint{0.5*\hexagonsize}{0mm}}
   \pgfpathlineto{\pgfpoint{\hexagonsize}{0.866025*\hexagonsize}}
   \pgfpathlineto{\pgfpoint{2*\hexagonsize}{0.866025*\hexagonsize}}
   \pgfpathlineto{\pgfpoint{2.5*\hexagonsize}{0mm}}
   \pgfusepath{stroke}
  }
%Exemple : \fill[pattern=hexagons] (0,0) circle (3cm);

\usetikzlibrary{patterns.meta}%Plus d'options de patterns

%%% Ajout de mylines :
\tikzdeclarepattern{
  name=mylines,
  parameters={
      \pgfkeysvalueof{/pgf/pattern keys/size},
      \pgfkeysvalueof{/pgf/pattern keys/angle},
      \pgfkeysvalueof{/pgf/pattern keys/line width},
  },
  bounding box={
    (0,-0.5*\pgfkeysvalueof{/pgf/pattern keys/line width}) and
    (\pgfkeysvalueof{/pgf/pattern keys/size},
     0.5*\pgfkeysvalueof{/pgf/pattern keys/line width})},
  tile size={(\pgfkeysvalueof{/pgf/pattern keys/size},
              \pgfkeysvalueof{/pgf/pattern keys/size})},
  tile transformation={rotate=\pgfkeysvalueof{/pgf/pattern keys/angle}},
  defaults={
    size/.initial=5pt,
    angle/.initial=45,
    line width/.initial=.4pt,
}, code={
\draw [line width=\pgfkeysvalueof{/pgf/pattern keys/line width}] (0,0) -- (\pgfkeysvalueof{/pgf/pattern keys/size},0);
}, }
%Exemple : \draw[pattern={mylines[size=10pt,line width=.8pt,angle=10]},pattern color=red] (0,0) rectangle ++(2,2);

%%% Ajout de hatch:
\pgfdeclarepattern{
name=hatch,
parameters={\hatchsize,\hatchangle,\hatchlinewidth},
bottom left={\pgfpoint{-.1pt}{-.1pt}},
top right={\pgfpoint{\hatchsize+.1pt}{\hatchsize+.1pt}},
tile size={\pgfpoint{\hatchsize}{\hatchsize}},
tile transformation={\pgftransformrotate{\hatchangle}},
code={
\pgfsetlinewidth{\hatchlinewidth}
\pgfpathmoveto{\pgfpoint{-.1pt}{-.1pt}}
\pgfpathlineto{\pgfpoint{\hatchsize+.1pt}{\hatchsize+.1pt}}
\pgfpathmoveto{\pgfpoint{-.1pt}{\hatchsize+.1pt}}
\pgfpathlineto{\pgfpoint{\hatchsize+.1pt}{-.1pt}}
\pgfusepath{stroke}
}
}
\tikzset{
hatch size/.store in=\hatchsize,
hatch angle/.store in=\hatchangle,
hatch line width/.store in=\hatchlinewidth,
hatch size=5pt,
hatch angle=0pt,
hatch line width=.5pt,
}
%Exemple : \filldraw[pattern color=gray!70,pattern=hatch,hatch size=10pt](3.2,0) rectangle (3.8,12)

\newtcbtheorem{theo}%
  {Théorème}{colback=red!5!white,colframe=red!50!black,title=My nice heading}{theorem}
%\begin{theo}{Like and Subscribe}{likesub}
%For every person $x$, $x$ should like and subscribe.
%\end{theo}
%\newcommand{\degree}{\ensuremath{^\circ}}
%\DeclareMathAccent{\wideparen}{\mathord}{largesymbols}{"F3}
%\def\wideparen#1{\overset{\;\rotatebox{90}{)}}{#1}}

%----------------------------------------

% \newcommand{\rul}[1]{\textcolor{red}{\underbar{\textcolor{black}{#1}}}}
% \newcommand{\bul}[1]{\textcolor{blue}{\underbar{\textcolor{black}{#1}}}}
% \newcommand{\vul}[1]{\textcolor{violet}{\underbar{\textcolor{black}{#1}}}}
% \newcommand{\gul}[1]{\textcolor{green!60!black}{\underbar{\textcolor{black}{#1}}}}
% \newcommand{\nul}[1]{\underbar{#1}}
% \newcommand{\attention}{\includegraphics[scale=0.2]{images/attention.png}\textcolor{red}{Attention} : }

% \newcommand{\rul}[1]{\textbf{\textsc{#1}}}
% \newcommand{\bul}[1]{\textsc{#1}}
% \newcommand{\gul}[1]{\textbf{#1}}
% \newcommand{\vul}[1]{\textsc{\textit{#1}}}

%%%%%%%%%%%% Pour avoir l'arc pour les plus que demi-cercle
\makeatletter
\def\widebreve{\mathpalette\wide@breve}
\def\wide@breve#1#2{\sbox\z@{$#1#2$}%
     \mathop{\vbox{\m@th\ialign{##\crcr
\kern0.08em\brevefill#1{0.8\wd\z@}\crcr\noalign{\nointerlineskip}%
                    $\hss#1#2\hss$\crcr}}}\limits}
\def\brevefill#1#2{$\m@th\sbox\tw@{$#1($}%
  \hss\resizebox{#2}{\wd\tw@}{\rotatebox[origin=c]{90}{\upshape(}}\hss$}
\makeatletter

%%%%%%%%%%% Pour avoir l'arc pour les moints que demi-cercle
\makeatletter
\def\wideparen{\mathpalette\wide@paren}
\def\wide@paren#1#2{\sbox\z@{$#1#2$}%
     \mathop{\vbox{\m@th\ialign{##\crcr
\kern0.08em\parenfill#1{0.8\wd\z@}\crcr\noalign{\nointerlineskip}%
                    $\hss#1#2\hss$\crcr}}}\limits}
\def\parenfill#1#2{$\m@th\sbox\tw@{$#1($}%
  \hss\resizebox{#2}{\wd\tw@}{\rotatebox[origin=c]{90}{\upshape)}}\hss$}
\makeatletter
%%%%%%%%%%% Pour avoir le chapeaux de angles rentrants
\makeatletter
\DeclareRobustCommand\widecheck[1]{{\mathpalette\@widecheck{#1}}}
\def\@widecheck#1#2{%
    \setbox\z@\hbox{\m@th$#1#2$}%
    \setbox\tw@\hbox{\m@th$#1%
       \widehat{%
          \vrule\@width\z@\@height\ht\z@
          \vrule\@height\z@\@width\wd\z@}$}%
    \dp\tw@-\ht\z@
    \@tempdima\ht\z@ \advance\@tempdima2\ht\tw@ \divide\@tempdima\thr@@
    \setbox\tw@\hbox{%
       \raise\@tempdima\hbox{\scalebox{1}[-1]{\lower\@tempdima\box
\tw@}}}%
    {\ooalign{\box\tw@ \cr \box\z@}}}
\makeatother
%%%%%%%%%%%%%%%%%%% Arc de cercle centré en 2 %%%%% Fonctionnent pas et je ne sais pas pourquoi
%\def\angle[#1](#2)(#3:#4:#5){\filldraw [#1,domain=#3:#4, fill opacity=0.2] (#2)--plot ((#2)+({#5*cos(\x)},{#5*sin(\x)})--(#2);}
%\filldraw [green!70!black,domain=105:435, fill opacity=0.2] (0,0) -- plot ({0.8*cos(\x)}, {0.8*sin(\x)})--(0,0);
%\angle[green!70!black](0,0)(105:435:0.8);
%\def\centerarc[#1](#2)(#3:#4:#5){\draw[#1]([shift=(#3:#5)]#2) arc (#3:#4:#5);}%
%Syntax: [draw options] (center) (initial angle:final angle:radius)
    
%\def\centerarc[#1](#2)(#3:#4:#5){ \draw[#1] ($(#2)+({#5*cos(#3)},{#5*sin(#3)})$) arc (#3:#4:#5) -- (#2); }
\def\angle[#1](#2)(#3,#4,#5){\filldraw [domain=#3:#4, fill opacity=0.2,#1] (#2)--($(#2)+({#5*cos(#3)},{#5*sin(#3)})$) arc (#3:#4:#5)--(#2);}
\def\centerarc[#1](#2)(#3,#4,#5){\draw[#1] ($(#2)+({#5*cos(#3)},{#5*sin(#3)})$) arc (#3:#4:#5); }
%{\draw [#1,domain=#3:#4] plot ((#2)+({#5*cos(\x)}, {#5*sin(\x)}));}
    % Syntax: [draw options] (center) (initial angle:final angle:radius)
%Exemple : \centerarc[red,thick](0,0)(5:85:1)
%---------------------- Pour les flèches pour les tableaux de proportionnalité
\newcommand{\marqueur}[1]{\tikz[remember picture, overlay]\node[xshift=-0.5ex,yshift=0.4ex] (#1){} ;}
\newcommand{\markh}[1]{\tikz[remember picture, overlay]\node[xshift=-0.5ex, yshift=0.5em] (#1){} ;}
\newcommand{\markb}[1]{\tikz[remember picture, overlay]\node[xshift=-0.5ex] (#1){} ;}

\newcommand{\fleche}[5]{\tikz [remember picture, overlay]\draw (#1) edge[min distance=1.5em,bend right=#3,-stealth] node[midway, #4]{#5} (#2);}
%----------------------------------------------------------
%---Convention française de 3D :
% \tikzset{math3d/.style=
% {x= {(-0.353cm,-0.353cm)}, z={(0cm,1cm)},y={(1cm,0cm)}}}
%----%--------------Pour les macros

% \title{\Huge \textcolor{red}{\'Evaluation de sixième}}
% \author{Réginald Bayle}
% \date{ }

\pagestyle{empty}
\setlength{\parskip}{8pt}

\begin{document}%\TeacherModeOn

\textbf{Nom et prénom} : \hfill \underbar{\textbf{Contrôle n°3 -- $6^{\text{ème}} 8$ Turquoise}} \hfill \phantom{Nom et prénom :}%\\[10pt]

\dline

\vspace{2.3cm}
\rule{\textwidth}{.5pt}

\emph{Chaque figure est à réaliser au crayon à papier. \underbar{Laisser apparents les traits de construction et coder.}}

\textbf{Exercice n°1 :} Effectuer les conversions suivantes sur l'énoncé : \hfill \textbf{/2 points}

34,5 km = \gap{34 400} m ; \quad 7 cm = \gap{0,07} m ; \quad 48 dm = \gap{4,8} m ; \quad 95 hm = \gap{9 500} m.

2300 m = \gap{230} dam ; \ 1700 dam = \gap{17} km ; \ 790 m = \gap{0,79} km ; \ 85 dm = \gap{8 500} mm

\bigskip\textbf{Exercice n°2 :} Observer la figure (unité : cm) et compléter : \hfill \textbf{/3 points}

\begin{minipage}{.59\linewidth}
    1. La distance du point $B$ à la droite $(AC)$ est \gap{5 cm}

    \medskip 2. La distance du point $A$ à la droite \gap{$(BC)$} est 12 cm.

    \medskip 3. Le point \gap{$D$} est situé à 10,5 cm de la droite \gap{$(EF)$}

    \medskip 4. Le point \gap{$F$} est situé à \gap{6 cm} de la droite $(DE)$.
\end{minipage}
\begin{minipage}{0.4\linewidth}
    \begin{tikzpicture}[line cap=round,line join=round,>=triangle 45,scale=.27,font=\small]
        \clip(-0.3,-3.6) rectangle (27.9,5.6);
        \draw[] (10.490093780402923,2.5242393534397887) -- (10.773287744633242,1.3473575974478647) -- (11.950169500625167,1.6305515616781858) -- (11.666975536394846,2.8074333176701094) -- cycle;
        \draw[] (26.882749706497023,-0.8603707494339998) -- (25.67227480425282,-0.8603707494339996) -- (25.67227480425282,-2.070845651678205) -- (26.882749706497023,-2.070845651678205) -- cycle;
        \draw[] (17.43373623908433,-1.470281918771173) -- (16.8331725061773,-0.4192953861838664) -- (15.782185973589991,-1.0198591190908985) -- (16.382749706497023,-2.070845651678205) -- cycle;
        \draw [] (0.,0.)-- (12.836739418757391,-2.053806489161076);
        \draw [] (0.,0.)-- (11.666975536394846,2.8074333176701094);
        \draw [] (11.666975536394846,2.8074333176701094)-- (12.836739418757391,-2.053806489161076);
        \draw (4,3.8) node[anchor=north west] {12};
        \draw (5.312656886790214,-1.0437236374786043) node[anchor=north west] {13};
        \draw (10.5,1.2) node[anchor=north west] {5};
        \draw [] (16.382749706497023,-2.070845651678205)-- (26.882749706497023,-2.070845651678205);
        \draw [] (26.882749706497023,3.929154348321792)-- (26.882749706497023,-2.070845651678205);
        \draw [] (26.882749706497023,3.929154348321792)-- (16.382749706497023,-2.070845651678205);
        \draw [] (13.405916076356018,3.1386132010685577)-- (16.382749706497023,-2.070845651678205);
        \draw [] (13.405916076356018,3.1386132010685577)-- (26.882749706497023,3.929154348321792);
        \draw (26.6,1.7) node[anchor=north west] {6};
        \draw (20.4,-1.8) node[anchor=north west] {10,5};
        \draw (18.8,5.4) node[anchor=north west] {13,5};
        % \begin{scriptsize}
        % \draw [color=black] (0.,0.)-- ++(-2.5pt,-2.5pt) -- ++(5.0pt,5.0pt) ++(-5.0pt,0) -- ++(5.0pt,-5.0pt);
        \draw[color=black] (0.4052961522810101,1.039051557981697) node {$A$};
        % \draw [color=black] (12.836739418757391,-2.053806489161076)-- ++(-2.5pt,-2.5pt) -- ++(5.0pt,5.0pt) ++(-5.0pt,0) -- ++(5.0pt,-5.0pt);
        \draw[color=black] (13.5,-2.1) node {$B$};
        % \draw [color=black] (11.666975536394846,2.8074333176701094)-- ++(-2.5pt,-2.5pt) -- ++(5.0pt,5.0pt) ++(-5.0pt,0) -- ++(5.0pt,-5.0pt);
        \draw[color=black] (11.589513640232221,4.006292932336098) node {$C$};
        % \draw [color=black] (16.382749706497023,-2.070845651678205)-- ++(-2.5pt,-2.5pt) -- ++(5.0pt,5.0pt) ++(-5.0pt,0) -- ++(5.0pt,-5.0pt);
        \draw[color=black] (15.5,-2.4) node {$D$};
        % \draw [color=black] (26.882749706497023,-2.070845651678205)-- ++(-2.5pt,-2.5pt) -- ++(5.0pt,5.0pt) ++(-5.0pt,0) -- ++(5.0pt,-5.0pt);
        \draw[color=black] (27.4,-2) node {$E$};
        % \draw [color=black] (26.882749706497023,3.929154348321792)-- ++(-2.5pt,-2.5pt) -- ++(5.0pt,5.0pt) ++(-5.0pt,0) -- ++(5.0pt,-5.0pt);
        \draw[color=black] (27.2,4.8) node {$F$};
        % \draw [color=black] (13.405916076356018,3.1386132010685577)-- ++(-2.5pt,-2.5pt) -- ++(5.0pt,5.0pt) ++(-5.0pt,0) -- ++(5.0pt,-5.0pt);
        \draw[color=black] (13.814944670998022,4.177479934702699) node {$G$};
        % \end{scriptsize}
    \end{tikzpicture}
\end{minipage}

Bonus : La distance du point $F$ à la droite $(DG)$ est comprise entre \gap{10,5 cm} et \gap{13,5 cm}

\begin{minipage}{.6\linewidth}
    \bigskip\textbf{Exercice n°3 :} Donner les distances :  \hfill \textbf{/2 points}

    {\small \emph{En indiquant sur la figure comment vous faites}}

    \medskip 1) entre les points $A$ et $E$ ? \gap{1,5 cm}

    \medskip\phantom{1)} entre les points $A$ et $F$ ? \gap{1,5 cm}

    \medskip 2) entre les droites $(d)$ et $(d')$ ? \gap{2,5 cm}

    \medskip 3) entre le point $E$ et la droite $(d)$ ? \gap{1,3 cm}
\end{minipage}\hfill
\begin{minipage}{.335\linewidth}
    \definecolor{qqwuqq}{rgb}{0.,0.39215686274509803,0.}
    \definecolor{cqcqcq}{rgb}{0.7529411764705882,0.7529411764705882,0.7529411764705882}
    \begin{tikzpicture}[line cap=round,line join=round,>=Latex,scale=.5]
        \draw [color=cqcqcq,, xstep=1.0cm,ystep=1.0cm] (-3.1,-1.5) grid (9.6,5.3);
        \clip(-3.1,-1.5) rectangle (9.6,5.3);
        % \draw[line width=0.8pt,color=qqwuqq] (4.254558441227157,-0.6605887450304574) -- (3.9151471862576144,-0.4060303038033003) -- (3.6605887450304575,-0.745441558772843) -- (4.,-1.) -- cycle; 
        % \draw[line width=0.8pt,color=qqwuqq] (5.334558441227157,0.7794112549695427) -- (4.995147186257614,1.0339696961966998) -- (4.740588745030458,0.694558441227157) -- (5.08,0.44) -- cycle; 
        \draw [line width=0.8pt] (0.,2.) circle (3.cm);
        \draw [line width=0.8pt,domain=-3.1:9.6] plot(\x,{(--6.--4.*\x)/3.});
        \draw [line width=0.8pt,domain=-3.1:9.6] plot(\x,{(-19.--4.*\x)/3.});
        \draw [<->,line width=0.4pt] (5.,4.) --node[above]{1 cm} (7.,4.);
        \draw (2.2,5.6) node[anchor=north west] {$(d)$};
        \draw (8,4.9) node[anchor=north west] {$(d')$};
        % \draw [line width=0.8pt,dash pattern=on 5pt off 5pt] (0.,2.)-- (3.,2.);
        % \draw [line width=0.8pt,dash pattern=on 5pt off 5pt] (1.465,2.09) -- (1.465,1.91);
        % \draw [line width=0.8pt,dash pattern=on 5pt off 5pt] (1.535,2.09) -- (1.535,1.91);
        % \draw [line width=0.8pt,dash pattern=on 5pt off 5pt] (0.,2.)-- (-2.487003576284264,0.3222594921891879);
        % \draw [line width=0.8pt,dash pattern=on 5pt off 5pt] (-1.1641545311844914,1.1060932780638593) -- (-1.2648189616531396,1.2553134926409149);
        % \draw [line width=0.8pt,dash pattern=on 5pt off 5pt] (-1.2221846146311235,1.066945999548273) -- (-1.3228490450997719,1.2161662141253287);
        % \draw [line width=0.8pt,dash pattern=on 5pt off 5pt] (0.,2.)-- (4.,-1.);
        % \draw [line width=0.8pt,dash pattern=on 5pt off 5pt] (3.,2.)-- (5.08,0.44);
        % \begin{scriptsize}
        \draw [color=black] (0.,2.)-- ++(-2.5pt,-2.5pt) -- ++(5.0pt,5.0pt) ++(-5.0pt,0) -- ++(5.0pt,-5.0pt);
        \draw[color=black] (-0.2,2.5) node {$A$};
        \draw [color=black] (3.,2.)-- ++(-2.5pt,-2.5pt) -- ++(5.0pt,5.0pt) ++(-5.0pt,0) -- ++(5.0pt,-5.0pt);
        \draw[color=black] (3.3,2.4) node {$E$};
        \draw [color=black] (7.,3.)-- ++(-2.5pt,-2.5pt) -- ++(5.0pt,5.0pt) ++(-5.0pt,0) -- ++(5.0pt,-5.0pt);
        \draw[color=black] (7.4,3.0) node {$C$};
        \draw [color=black] (-2.487003576284264,0.3222594921891879)node[below,xshift=-2pt] {$F$}-- ++(-2.5pt,-2.5pt) -- ++(5.0pt,5.0pt) ++(-5.0pt,0) -- ++(5.0pt,-5.0pt);
        % \draw[color=black] (-2.52,0.81) node {$F$};
        % \end{scriptsize}
    \end{tikzpicture}
\end{minipage}


\bigskip\textbf{Exercice n°4 :}  \hfill \textbf{/3 points}\\
\begin{minipage}{0.5\linewidth}
    Dans la figure ci-contre,

    \bigskip $\bullet$ $\mathcal{C}_1$ est un cercle de centre $A$ et de rayon 9 cm.

    \medskip$\bullet$ $\mathcal{C}_2$ est un cercle de centre $B$ de diamètre 20 cm.

    \medskip$\bullet$ $AB=12$ cm.

    \medskip$\bullet$ $E,\ A,\ D,\ R,\ B$ et $C$ sont alignés.

    \bigskip 1) Donner les longueurs suivantes
\end{minipage}
\begin{minipage}{0.5\linewidth}
    \begin{tikzpicture}[line cap=round,line join=round,>=triangle 45,scale=0.25]
        \clip(-9.3,-10.2) rectangle (22.6,10.1);
        \draw [] (0.,0.) circle (9.cm);
        \draw [] (12.,0.) circle (10.cm);
        \draw (-9,8.8) node[anchor=north west] {$(\mathcal{C}_1)$};
        \draw (18.385535696461925,9.371103193094422) node[anchor=north west] {$(\mathcal{C}_2)$};
        \draw [] (-9.,0.)-- (22.,0.);
        % \begin{scriptsize}
        \draw [color=black] (0.,0.)node{$\times$};%-- ++(-2.5pt,-2.5pt) -- ++(5.0pt,5.0pt) ++(-5.0pt,0) -- ++(5.0pt,-5.0pt);
        \draw[color=black] (0.34507854445622727,0.9026439182832428) node {$A$};
        \draw [color=black] (12.,0.)node{$\times$};%-- ++(-2.5pt,-2.5pt) -- ++(5.0pt,5.0pt) ++(-5.0pt,0) -- ++(5.0pt,-5.0pt);
        \draw[color=black] (12.340063352970656,0.9026439182832428) node {$B$};
        % \draw [color=black] (22.,0.)node{$\times$};%-- ++(-2.5pt,-2.5pt) -- ++(5.0pt,5.0pt) ++(-5.0pt,0) -- ++(5.0pt,-5.0pt);
        \draw[color=black] (21.2,1) node {$C$};
        % \draw [color=black] (2.,0.)node{$\times$};%-- ++(-2.5pt,-2.5pt) -- ++(5.0pt,5.0pt) ++(-5.0pt,0) -- ++(5.0pt,-5.0pt);
        \draw[color=black] (2.7,0.9) node {$D$};
        % \draw [color=black] (-9.,0.)node{$\times$};%-- ++(-2.5pt,-2.5pt) -- ++(5.0pt,5.0pt) ++(-5.0pt,0) -- ++(5.0pt,-5.0pt);
        \draw[color=black] (-8.3,1) node {$E$};
        % \draw [color=black] (5.208333333333333,7.339840862640613)node{$\times$};%-- ++(-2.5pt,-2.5pt) -- ++(5.0pt,5.0pt) ++(-5.0pt,0) -- ++(5.0pt,-5.0pt);
        \draw[color=black] (5.1,8.3) node {$M$};
        % \draw [color=black] (5.208333333333333,-7.339840862640613)node{$\times$};%-- ++(-2.5pt,-2.5pt) -- ++(5.0pt,5.0pt) ++(-5.0pt,0) -- ++(5.0pt,-5.0pt);
        \draw[color=black] (5.1,-8.4) node {$N$};
        % \draw [color=black] (9.,0.)node{$\times$};%-- ++(-2.5pt,-2.5pt) -- ++(5.0pt,5.0pt) ++(-5.0pt,0) -- ++(5.0pt,-5.0pt);
        \draw[color=black] (9.4,1) node {$R$};
        % \end{scriptsize}
    \end{tikzpicture}
\end{minipage}\\
$AM$ = \gap{9 cm} ; \quad $BN$ = \gap{10 cm} ; \quad $BD$ = \gap{10 cm} ; \quad $AR$ = \gap{9 cm} ; \quad $ER$ = \gap{18 cm}\\%\ $CD$ = \gap{20 cm} ;
2) Déterminant les longueurs suivantes en détaillant les calculs :

\qquad $BE =$ \gap{$BA+AE=12+9=21$ cm car $A\in[EB]$.}

\qquad $BR =$ \gap{$BA-AR=12-9=3$ cm car $R\in[AB]$.}

\qquad $AD =$ \gap{$AB-BD=12-10=2$ cm car $D\in[AB]$.}

\qquad $DR =$ \gap{$AR-AD=9-2=7$ cm car $D\in[AR]$.}

\begin{minipage}{.6\linewidth}
    \bigskip\textbf{Exercice n°5 :}  \hfill \textbf{/3 points}

    \medskip Placer les points $A,\ B,\ C,\ D,\ E,\ F,\ I$ ci-contre sachant que :

    \medskip$\bullet$ La corde $[DF]$ passe par le milieu $I$ du segment $[AC]$.

    \medskip$\bullet$ $C$ est le centre du cercle de diamètre $[AE]$.

    \medskip$\bullet$ $A$ est un point du segment $[BE]$.

    \medskip$\bullet$ $EF<DE$.
\end{minipage}
\hfill
\begin{minipage}{.3\linewidth}
    \begin{tikzpicture}[line cap=round,line join=round,>=triangle 45,scale=.7]
        \clip(-3.1,-3.1) rectangle (3.1,3.1);
        \draw [] (0.,0.) circle (3.cm);
        \draw [] (1.2588145896433243,-0.8157118540888744) circle (1.5cm);
        \draw [] (-2.0813059491230224,-2.1605937948038996)-- (2.8243707657156536,1.0113999099123805);
        \draw [] (0.,0.)-- (0.6294072948216621,-0.4078559270444372);
        \draw [] (0.3380001094164966,-0.07603240121445705) -- (0.2074862127622766,-0.2774427355573884);
        \draw [] (0.4219210820593851,-0.13041319148704866) -- (0.29140718540516514,-0.33182352582998004);
        \draw [] (0.6294072948216621,-0.4078559270444372)-- (1.2588145896433243,-0.8157118540888744);
        \draw [] (0.9674074042381583,-0.4838883282588941) -- (0.8368935075839383,-0.6852986626018255);
        \draw [] (1.0513283768810469,-0.5382691185314857) -- (0.9208144802268269,-0.7396794528744172);
        \draw [] (-2.5176291792866485,1.6314237081777487)-- (0.,0.);
        \draw [] (1.2588145896433243,-0.8157118540888744)-- (2.5176291792866485,-1.6314237081777487);
        % \begin{scriptsize}
        \draw [color=black] (0.,0.) node{$\bullet$};
        % \draw[color=black] (-0.28,0.07) node {$A$};
        \draw [color=black] (-2.5176291792866485,1.6314237081777487)node{$\bullet$};%-- ++(-2.5pt,-2.5pt) -- ++(5.0pt,5.0pt) ++(-5.0pt,0) -- ++(5.0pt,-5.0pt);
        % \draw[color=black] (-2.66,2.01) node {$B$};
        \draw [color=black] (2.5176291792866485,-1.6314237081777487)node{$\bullet$};%-- ++(-2.5pt,-2.5pt) -- ++(5.0pt,5.0pt) ++(-5.0pt,0) -- ++(5.0pt,-5.0pt);
        % \draw[color=black] (2.76,-1.49) node {$E$};
        \draw [color=black] (1.2588145896433243,-0.8157118540888744)node{$\bullet$};%-- ++(-2.5pt,-2.5pt) -- ++(5.0pt,5.0pt) ++(-5.0pt,0) -- ++(5.0pt,-5.0pt);
        % \draw[color=black] (1.44,-0.57) node {$C$};
        \draw [color=black] (0.6294072948216621,-0.4078559270444372)node{$\bullet$};%-- ++(-2.5pt,-2.5pt) -- ++(5.0pt,5.0pt) ++(-5.0pt,0) -- ++(5.0pt,-5.0pt);
        % \draw[color=black] (0.76,-0.03) node {$I$};
        \draw [color=black] (-2.0813059491230224,-2.1605937948038996)node{$\bullet$};%-- ++(-2.5pt,-2.5pt) -- ++(5.0pt,5.0pt) ++(-5.0pt,0) -- ++(5.0pt,-5.0pt);
        % \draw[color=black] (-2.48,-2.15) node {$D$};
        \draw [color=black] (2.8243707657156536,1.0113999099123805)node{$\bullet$};%-- ++(-2.5pt,-2.5pt) -- ++(5.0pt,5.0pt) ++(-5.0pt,0) -- ++(5.0pt,-5.0pt);
        % \draw[color=black] (2.96,1.39) node {$F$};
        % \end{scriptsize}
    \end{tikzpicture}
\end{minipage}

\bigskip\textbf{Exercice n°6 :} \hspace{4 cm} \textbf{/5 points}\\
\emph{Figure à faire dans l'espace à droite.}\\
1) Tracer un segment $[RS]$ de longueur 4,9 cm.\\
2) Construire sa médiatrice $(d)$.\\
3) Placer un point $E$ de $(d)$ tel que : $SE=3$ cm.\\
4) Que peut-on dire de la longueur $ER$ ? Justifier.

\dotfill

\dotfill

\dotfill

\dotfill\\
5) Tracer $(d')$ la perpendiculaire à $(d)$ passant par $E$.\\
\phantom{5)} Que peut-on alors dire de $(RS)$ et $(d')$ ? justifier

\dotfill

\dotfill

\dotfill

\dotfill

%\begin{minipage}{.5\linewidth}
%    \begin{tikzpicture}
%        
%    
%    \end{tikzpicture}
%\end{minipage}

\bigskip\textbf{Exercice n°7 :} \hfill \textbf{/3 points}\\
Marquer sur la figure l'ensemble de tous les points qui sont à moins de 3,5 cm du point $A$, à égale distance de $A$ et $B$, et à plus de 1 cm de la droite $(AB)$. \hfill \emph{Indiquer les légendes.}\\
\begin{tikzpicture}[line cap=round,line join=round,>=triangle 45,x=1.0cm,y=1.0cm]
    \clip(-1.4,-2.8) rectangle (9.,4.3);
    % \draw[] (5.668889663098127,0.028264493313237982) -- (5.647189111144779,0.23928365368717197) -- (5.436169950770845,0.21758310173382417) -- (5.457870502724193,0.006563941359890166) -- cycle; 
    % \draw[] (5.377273671732853,0.7902986427239582) -- (5.588292832106787,0.8119991946773059) -- (5.566592280153439,1.02301835505124) -- (5.355573119779505,1.0013178030978922) -- cycle; 
    % \draw[] (5.274976288788165,1.7850525044619607) -- (5.485995449162099,1.8067530564153085) -- (5.464294897208751,2.0177722167892425) -- (5.253275736834817,1.9960716648358947) -- cycle; 
    % \draw [,dash pattern=on 5pt off 5pt] (2.62,0.72) circle (3.5cm);
    % \draw [,dash pattern=on 5pt off 5pt,domain=-1.4:9.] plot(\x,{(-29.864500784105573--5.471146239559011*\x)/-0.5626356061957847});
    % % \draw [] (5.355573119779505,1.0013178030978922) circle (1.cm);
    % \draw [,dash pattern=on 5pt off 5pt,domain=-1.4:9.] plot(\x,{(-7.965120004249535-0.5626356061957847*\x)/-5.471146239559011});
    % \draw [,dash pattern=on 5pt off 5pt,domain=-1.4:9.] plot(\x,{(--3.0348799957504693-0.5626356061957847*\x)/-5.471146239559011});
    % \draw [] (2.62,0.72)-- (5.355573119779505,1.0013178030978922);
    % \draw [] (3.9257731808494887,0.9749144958102717) -- (3.950324552756214,0.7361735689931512);
    % \draw [] (4.02524856702329,0.9851442341047404) -- (4.049799938930016,0.7464033072876199);
    % \draw [] (5.355573119779505,1.0013178030978922)-- (8.091146239559011,1.2826356061957846);
    % \draw [] (6.6613463006289955,1.256232298908164) -- (6.685897672535721,1.0174913720910437);
    % \draw [] (6.760821686802797,1.2664620372026327) -- (6.7853730587095225,1.0277211103855124);
    % \draw [line width=2.8pt] (5.134092788852593,3.1550230900423504)-- (5.253275736834817,1.9960716648358947);
    % \draw [line width=2.8pt] (5.457870502724193,0.006563941359890166)-- (5.577053450706417,-1.1523874838465655);
    \draw [color=black] (2.62,0.72)-- ++(-2.5pt,-2.5pt) -- ++(5.0pt,5.0pt) ++(-5.0pt,0) -- ++(5.0pt,-5.0pt);
    \draw[color=black] (2.76,1.09) node {$A$};
    \draw [color=black] (8.091146239559011,1.2826356061957846)-- ++(-2.5pt,-2.5pt) -- ++(5.0pt,5.0pt) ++(-5.0pt,0) -- ++(5.0pt,-5.0pt);
    \draw[color=black] (8.24,1.65) node {$B$};
\end{tikzpicture}\\
\begin{minipage}{0.6\linewidth}
    \textbf{Exercice bonus :}  \hspace{4cm} \textbf{/2 points}

    Tracer l'ensemble des points situés à 1,5 cm du segment.
\end{minipage}
\begin{minipage}{0.35\linewidth}\centering
    \begin{tikzpicture}[line cap=round,line join=round,>=triangle 45,x=1.0cm,y=1.0cm]
        \clip(-0.5,0.4) rectangle (5.6,3.6);
        \draw [] (1.,2.)-- (4.,2.);
        % \draw [dash pattern=on 5pt off 5pt] (1.,3.5)-- (4.,3.5);
        % \draw [dash pattern=on 5pt off 5pt] (1.,0.5)-- (4.,0.5);
        % \draw [shift={(4.,2.)},dash pattern=on 5pt off 5pt]  plot[domain=-1.5707963267948966:1.5707963267948966,variable=\t]({1.*1.5*cos(\t r)+0.*1.5*sin(\t r)},{0.*1.5*cos(\t r)+1.*1.5*sin(\t r)});
        % \draw [shift={(1.,2.)},dash pattern=on 5pt off 5pt]  plot[domain=1.5707963267948966:4.71238898038469,variable=\t]({1.*1.5*cos(\t r)+0.*1.5*sin(\t r)},{0.*1.5*cos(\t r)+1.*1.5*sin(\t r)});
    \end{tikzpicture}
\end{minipage}
\end{document}


Dans son jardin, Romain a un poulailler rectangulaire de 4 m sur 3 m, construit avec des poteaux régulièrement espacés.
Il attache son âne Tintin avec une corde de 9 m à l'un des poteaux du poulailler, comme sur le schéma.
Sachant que Tintin ne peut pas entrer dans le poulailler, colorier la partie du jardin qu'il pourra brouter en représentant 1 m de terrain par 1 cm sur le papier.
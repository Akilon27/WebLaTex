\usepackage[utf8]{inputenc}
\usepackage[T1]{fontenc}
\usepackage[french]{babel}
\frenchbsetup{StandardItemLabels} % pour conserver les puces lors des énumérations
%\usepackage[babel=true,kerning=true]{microtype}
\usepackage[margin=1cm]{geometry}  %permet le réglage des marges
%\geometry{hmargin=1cm,vmargin=1cm} %réglage des marges

%%%%%%%%%%%%%%%%%%%%%%%%%%%%%%%%
% \usepackage{sansmath}
% \sansmath

%%%%%%%%%%%%%%%%%%%%%%%%
%---------Packages MathPix


%\usepackage[export]{adjustbox} %This package allows to adjust general (LA)TEX material in several ways using a key=value interface. It got inspired by the interface of \includegraphics from the graphicx package. This package also loads the trimclip package which code was once included in this package.
\usepackage{bm}%Pour avoir des formules mathématiques en gras
%\usepackage[version=4]{mhchem}
%\usepackage{stmaryrd}
%--------------------------

\usepackage{parskip} %évite que chaque nouvelle ligne soit indentée
\usepackage{setspace}
%\usepackage{xcolor}%[usenames, dvipsnames]
\usepackage{color,colortbl}
\usepackage[table]{xcolor}
%\usepackage{xspace}%va détecter lors de la compilation si une espace doit être ajoutée ou pas à la fin de la macro. On saisira donc : \newcommand\IP{imaginaires purs\xspace}
%\usepackage{picins}%Pour mettre facilement des images à droite
\usepackage{wrapfig}%Un peu la même chose
\setlength{\columnsep}{1cm}
\usepackage[overload]{textcase} 
\usepackage{dashundergaps} %Pour faire des cours à trous
\dashundergapssetup{dot,teacher-gap-format=dot,gap-format-adjust=false,
gap-numbers=false,gap-widen,gap-extend-minimum=30pt,%ajoute x pt au minimum
gap-extend-percent=40,%ajoute x % The result is compared to gap-extend-minimum and the larger of the two is used
%gap-font=\Large,%\bfseries \itshape 
%gap-number-format=\textnormal{ ^{(\thegapnumber)}},
%gap-number-format = \,\textsuperscript{\normalfont(\thegapnumber)},
%display-total-gaps,
}

\usepackage{siunitx}
\sisetup{locale = FR}
%%% Donne accès à la commande \num
%Exemples : \num{.34}   \num{3e8}

%\usepackage[np]{numprint}  %Autre paquet qui gère les espaces dans l'écriture des nombres

%\abovedisplayskip=1ex    Pour changer l'espacement au-dessus des formules mathématiques \[ \]... Il y a aussi \belowdisplayskip ainsi que \abovedisplayshortskip et  \belowdisplayshortskip si la ligne qui PRECEDE est trop courte !

%Assez des salamis, je passe au jambon - Je fais un carnage si ce car nage car je nage, moi, Karnaj !
%Le comble pour un professeur de mathématique ? Mourir dans l’exercice de ses fonctions.

%$\circlearrowleft$ $\circlearrowright$
%$\curvearrowleft$ $\curvearrowright$

\usepackage{cancel}
%$x^2-\cancel{x}+\cancel{x}-1=x^2-1$        $\cancel{A}+\bcancel{B}+\xcancel{C}
%---------------------------
%%%%%%%%%%%%%%%%%%%%%%%%%%%%%%%%%%%%%%%%%%%%%%
\usepackage{amsmath, amsthm, amssymb} % nécessaire pour taper des maths

%%%%%%%%%%%%%%%%%%%%%%%%%%%%%%%%%%%%%%%%%%%%%%%%%%%%%%%
%\usepackage{lscape}  % pour le format paysage
\usepackage{pifont}  % pour les symboles ding
\usepackage{enumerate}  %Permet de renuméroter différement les questions
%\usepackage{esvect}  % permet de faire des beaux vecteurs
\usepackage{multicol}  % permet de fusionner des colonnes
\usepackage{multirow}   % permet de fusionner des lignes
%\usepackage{colortbl}  %couleur dans les tableaux
\usepackage{tabularx}  % permet des tableaux plus compliqués mais pas simple à utiliser.
\usepackage{diagbox}%---Des cellules en diagonales pour tableaux à doubles entrée
\usepackage{array} %,makecell
% \newcolumntype{R}[1]{>{\raggedleft\arraybackslash }b{#1}}  % permet des cellules alignées à droite de taille fixée : écrire L{3cm} par exemple. Ne convient pas à une fusion
% \newcolumntype{L}[1]{>{\raggedright\arraybackslash }b{#1}} % même chose à gauche
\newcolumntype{C}[1]{ >{\centering\arraybackslash} m{#1}}   % même chose au centre
\newcolumntype{T}[1]{ >{\centering\arraybackslash\columncolor{teal!20}} m{#1}} %Avec couleur teal pour les titres de lignes

%\setcellgapes{1pt}     %ajouter  \setcellgapes{4pt} avant un tableau avec des formules mathématiques hautes pour que cela ne touche pas le bord de la cellule
%\makegapedcells
\usepackage{graphicx} % pour l'insersion d'images, incompatible avec xcolor
%\usepackage{caption} % pour des titres dans les tableaux flottants, pas forcément utile
%\usepackage{variations} % permet des supers tableaux de variations
\usepackage{frcursive}  % permet d'écrire comme un écolier, c'est très joli
\everymath{\displaystyle} %pour que les maths soient à la bonne taille (limite, intégrale)
%%%%%%%%%%%%%%%%%%%GrindEQ
\usepackage{txfonts}
\usepackage{mathdots}
\usepackage[classicReIm]{kpfonts}
%%%%%%%%%%%%%%%%%%%%%%%%%%%%%%%

\usepackage{marvosym}   %c'est pour le symbol euro : code \EUR{}
%\usepackage{pst-3dplot}  % cela fait des dessins en 2D mais je ne maitrise pas bien

\usepackage{mathrsfs}
%\usepackage{yhmath}%Pour les arcs de cercles mais des conflits : mieux ci-dessous, redéfini
%\usepackage{tikz}
\usepackage{tkz-euclide}%,pgflibraryshapes
% \usepackage{pgfplots}
\usetikzlibrary{calc,through}
\usetikzlibrary{arrows}
\usetikzlibrary[patterns]
%Pattern names : horizontal lines, vertical lines, north east lines, north west lines, grid, crosshatch	dots, fivepointed stars, sixpointed stars, bricks, checkerboard

\usetikzlibrary{positioning}
%\usepackage{tkz-tab}
%\usepackage{forest}%Pour les arbres de probabilités
% \usetikzlibrary{shadows}%Pour les ombres Tikz
%\usepackage{tikz-qtree}% Autre paquet pour les arbres
%\usepackage{fancyhdr,txfonts,pxfonts}
\usepackage[most]{tcolorbox}
